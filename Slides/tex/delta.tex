%!TEX root = ../main.tex
%
\section[Delta robot]{Delta robot}
\begin{frame}
\frametitle{Delta robot}
The Delta robot is a 3-DOF parallel kinematic machine developed by Reymond Clavel~\footfullcite{Clavel1991} in 1991. It mainly consists of three actuated kinematic chains linked at a common moving platform. Each chain is a serial connection of a revolute actuator, a rear-arm and a forearm (composed of two parallel rods forming a parallelogram). The rear-arms and the forearms are linked through ball-and-socket passive joints. The parallelogram structure of the forearms ensures that the moving platform stays always parallel to the fixed base.
Figure~\ref{Delta_schematic} shows a schematic view of the Delta robot with its main elements highlighted.
\end{frame}
%
\begin{frame}
\frametitle{Delta robot - Schematic view}
\begin{columns}
    \begin{column}{0.5\textwidth}
		\vspace*{.6cm}
    	\begin{figure}
    		\frame{\includegraphics[width=.8\linewidth]{img/Delta.png}}
    		\caption{Schematic view of Delta robot}
    		\label{Delta_schematic}
    	\end{figure}
    \end{column}
    \begin{column}{0.5\textwidth}
        \begin{enumerate}
        	\item Fixed base-plate
        	\item Actuator
        	\item Rear-arm
        	\item Forearm
        	\item Moving platform
        \end{enumerate}
    \end{column}
\end{columns}
\end{frame}
%
\begin{frame}
\frametitle{Delta robot - Parameters}
Analytical studies on the working volume of the Delta robot~\footfullcite{Rey} demonstrated that:
\begin{itemize}
 	\item A ratio $r = R/l_A < 0.63$ gives the most regular shape for the surface of the lower part of the working volume, with $R$ as the distance between the center of the base plate and the rotation axis of the actuator and $l_A$ as the length of the rear-arm.
 	\item If $r > 0.0484$ and $b = l_A/l_B > 1.75$ there is no singularity occurrance within the robot working volume.
\end{itemize}
Thus the parameters shown in table~\ref{Delta_param} have been chosen for the Delta model used in this project.
\end{frame}
%
\begin{frame}
\frametitle{Delta robot - Parameters}
\begin{table}
\begin{center}
\begin{tabu} to \textwidth { | X[c] | X[c] | X[c] | }
	\hline
	\textbf{Parameter} & \textbf{Description} & \textbf{Value} \\
	\hline
	$l_A$ & Rear-arm length & $0.2 m$ \\
	$m_A$ & Rear-arm mass & $0.1 Kg$ \\
	$R$ & Base platform parameter & $0.126 m$ \\
	$l_B$ & Forearm length & $0.4 m$ \\
	$m_B$ & Forearm mass & $0.045 Kg$ \\
	$m_c$ & Elbow mass & $0.018 Kg$ \\
	$m_n$ & Moving platform mass & $0.1 Kg$ \\
	$I_{bi}$ & Rear-arm inertia & $Kg\times m^2$ \\
	\hline
\end{tabu}
\caption{Delta robot geometric and dynamic parameters}
\label{Delta_param}
\end{center}
\end{table}
\end{frame}
%
\subsection{Direct kinematic}
\begin{frame}
\frametitle{Delta robot - Direct kinematic}
	\begin{columns}
    \begin{column}{0.48\textwidth}
        	\begin{figure}
        		\includegraphics[width=1\linewidth]{img/DeltaRobotClavelpg26.JPG}
        	\end{figure}
    \end{column}
    \begin{column}{0.48\textwidth}
        %Content
    \end{column}
\end{columns}

	% \begin{textblock*}{8cm}(2cm,2cm) % {block width} (coords)
	% 	\includegraphics[width=.5\linewidth]{img/DeltaRobotClavelpg26.JPG}
	% \end{textblock*}

	% \begin{figure}
	% 	\includegraphics[width=.5\linewidth]{img/DeltaRobotClavelpg26.JPG}
	% \end{figure}
\end{frame}
%
% \begin{frame}{The minipage environment}
% \begin{minipage}{0.47\textwidth}
%     \begin{itemize}
%         \item First item
%         \item Second item
%         \item Third item
%     \end{itemize}
% \end{minipage}
% \begin{minipage}{0.5\textwidth}
%     \rule{\textwidth}{0.75\textwidth}
% \end{minipage}
% \end{frame}

\begin{frame}
\frametitle{Delta robot - Direct kinematic}
	$C_i$ coordinates are given by the intersection of three circles of radius $L_A$ belonging to the plane $\pi_i$ and the sphere centered in $P$ having radius $L_B$.
	\[
	C_i =%
	\begin{pmatrix}
		(R + L_Acos\alpha_i)cos\theta_i\\
		(R + L_Acos\alpha_i)sin\theta_i\\
		-L_Asin\alpha_i
	\end{pmatrix}
	\]
	%
	Equation of the sphere centered in P:
	\begin{equation}
		(x - p_x)^2 + (y - p_y)^2 + (z - p_z)^2 = L_B
	\end{equation}
	%
	\begin{equation}
		((R + L_Acos\alpha_i)cos\theta_i - x^2) + ((R + L_Acos\alpha_i)sin\theta_i - y)^2 + (L_Asin\alpha_i + z)^2 = L_B^2
	\end{equation}

\end{frame}
\begin{frame}
	\frametitle{Delta robot - Direct kinematic}
	\begin{align}
		D_i &= R^2+2\,\cos\,q_{i}\,R\,l_{A}+{l_{A}}^2-{l_{B}}^2\\
		E_i &= \cos\theta _{i}\,\left(2\,R+2\,l_{A}\,\cos\,q_{i}\right)\\
		F_i &= \sin\theta _{i}\,\left(2\,R+2\,l_{A}\,\cos\,q_{i}\right)\\
		G_i &= -2\,l_{A}\,\sin\left(q_{i}\right)
	\end{align}
%
	\begin{align}
		H_1 &= E_{1}\,G_{2}-E_{2}\,G_{1}-E_{1}\,G_{3}+E_{3}\,G_{1}+E_{2}\,G_{3}-E_{3}\,G_{2}\\
		H_2 &= E_{2}\,F_{1}-E_{1}\,F_{2}+E_{1}\,F_{3}-E_{3}\,F_{1}-E_{2}\,F_{3}+E_{3}\,G_{2}\\
		H_3 &= D_{1}\,E_{2}-D_{1}\,E_{1}-D_{1}\,E_{3}+D_{3}\,E_{1}+D_{2}\,E_{3}-D_{3}\,E_{2}\\
		H_4 &= D_{2}\,F_{1}-D_{1}\,F_{2}+D_{1}\,F_{3}-D_{3}\,F_{1}-D_{2}\,F_{3}+D_{3}\,F_{2}\\
		H_5 &= F_{2}\,G_{1}-F_{1}\,G_{2}+F_{1}\,G_{3}-F_{3}\,G_{1}-F_{2}\,G_{3}+F_{3}\,G_{2}
	\end{align}
\end{frame}
%
\begin{frame}
\frametitle{Delta robot - Direct kinematic}
	\begin{align}
		L &= \frac{{H_{1}}^2+{H_{5}}^2}{{H_{2}}^2}+1\\
		M &= G_{1}-\frac{E_{1}\,H_{5}+F_{1}\,H_{1}}{H_{2}}+\frac{2\,H_{1}\,H_{3}+2\,H_{4}\,H_{5}}{{H_{2}}^2}\\
		N &= D_{1}-\frac{E_{1}\,H_{4}+F_{1}\,H_{3}}{H_{2}}+\frac{2\,{H_{3}}^2+2\,{H_{4}}^2}{{H_{2}}^2}
	\end{align}
\end{frame}
%
\begin{frame}
\frametitle{Delta robot - Direct kinematic}
End effector coordinates computation:
	\begin{equation}
		z_{1,2} = -\frac{M \pm \sqrt{M^2-4\,L\,N}}{2\,L}
	\end{equation}
	%
	Among the two solutions we pick the one with lower height that belongs to the Delta robot workspace.
	%
	\begin{align}
		x &= \frac{H_{4}}{H_{2}}-\frac{H_{5}\,\left(M-\sqrt{M^2-4\,L\,N}\right)}{2\,H_{2}\,L}\\
		y &= \frac{H_{3}}{H_{2}}-\frac{H_{1}\,\left(M-\sqrt{M^2-4\,L\,N}\right)}{2\,H_{2}\,L}
	\end{align}
\end{frame}
%
\subsection{Inverse kinematic}
\begin{frame}
	\frametitle{Delta robot - Inverse kinematic}
	\begin{align}
		A &= {L_{A}}^2-{L_{B}}^2-R^2+{x_{i}}^2+{y_{i}}^2+{z_{i}}^2\\
		B &= 2x_{i}-2R
	\end{align}
	%
	\begin{equation}
		z = \frac{A - Bx}{2\,z_{i}}
	\end{equation}
	%
	where:
	\begin{equation}
		x = \frac{b+\sqrt{b^2-a\,c}}{a}
	\end{equation}
	with:
	\begin{align}
		a &= {\left(2\,R-2\,x_{i}\right)}^2+4\,{z_{i}}^2\\
		b &= 4\,R\,{z_{i}}^2 + A\,B\\
		c &= A^2 - 4\,{L_{A}}^2\,{z_{i}}^2+4\,R^2\,{z_{i}}^2
	\end{align}
	%
	\begin{equation}
		q_i = -\mathrm{asin}\left(\frac{z}{L_{A}}\right)
	\end{equation}
\end{frame}
%
\subsection{Jacobian}
\begin{frame}
	\frametitle{Delta robot - Jacobian computation}
	\begin{equation}
		\left(\begin{array}{c} p-R_{b}\,\left(\overline{R}-L_{A}\,\cos\left(\overline{q_{i}}\right)\right)\\ p\\ p+L_{A}\,R_{b}\,\sin\left(\overline{q_{i}}\right) \end{array}\right)
	\end{equation}
\end{frame}